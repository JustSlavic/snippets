\documentclass{report}

\usepackage[T1]{fontenc}
\usepackage[utf8]{inputenc}
\usepackage[english, russian]{babel} 
\usepackage[scaled = 0.8]{PTMono} %monotype font
% \usepackage{paratype} %fonts
\usepackage{indentfirst} %indent after sections
\usepackage{xspace} % for additional spaces

\usepackage[left = 2cm, right = 1cm, top = 2cm, bottom = 2cm]{geometry}

\usepackage{ltablex}  % for centering inside cells of tables
\usepackage{pgfplots} % for histograms
\usepackage{xcolor}   % for defining new colors
\usepackage{amsmath}  % for split in equations
\usepackage{amssymb}  % for >= <=
\renewcommand {\le} {\leqslant}
\renewcommand {\ge} {\geqslant}

\definecolor{dark-blue}{rgb}{0.108, 0.3137, 0.70196}
\definecolor{dark-green}{rgb}{0.108, 0.7, 0.3}

\usepackage{listings}
\lstset { basicstyle = \ttfamily
        , showspaces = false
        , showstringspaces = false
        }

\begin{document}


\begin{centering}
Отчёт

По индивидуальному заданию

Вариант 23, 24

Выполнил: Радько~В.\,А.\\
\end{centering}

\vspace{7mm}

\subsubsection*{Первичная обработка статистических данных}

Была дана выборка значений случайных величин <<Число вызовов, поступивших на АТС за некоторый промежуток времени>>, обозначим её $\xi$, и <<Продолжительность разговора (мин.)>> обозначим её $\eta$.

Рассмотрим первую выборку $\{ x_i \}$, её объём $n = 100$. 
Сначала определим размах выборки, для этого находим $\min(x_i) = 1$ и $\max(x_i) = 21$. 
Разобьём интервал $[1, 21]$ на $k_1 = 10$ частей и построим вариационный ряд.

\begin{table}[h]
\begin{tabularx}{\textwidth}{|c|c|c|c|c|c|c|c|c|c|c|}
\hline
    $\left [a_i, a_{i+1}\right)$ & $\left [1, 3\right )$ & $\left [3, 5\right )$ & $\left [5, 7\right )$ & $\left [7, 9\right )$ & $\left [9, 11\right )$ & $\left [11, 13\right )$ & $\left [13, 15\right )$ & $\left [15, 17\right )$ & $\left [17, 18\right )$ & $\left [19, 21\right ]$ \\
\hline
    $\frac{m_i}{n}$ & $0.06$ & $0.23$ & $0.35$ & $0.2$ & $0.1$ & $0.04$ & $0.01$ & $0.0$ & $0.0$ & $0.1$ \\
\hline
\end{tabularx}
\setcounter{table}{0}
\caption {Вариационный ряд $\{ x_i \}$}
\end{table}

\begin{figure}[h] \label{fig:1}
\centering
\begin{tikzpicture} [scale = 0.8]
    \begin{axis}[ height = 0.4\paperheight
                , width = 0.7\paperwidth
                % , extra y ticks={0,0.05,...,0.35}
                , xtick = data
                , ytick = data
                , yticklabel style = { /pgf/number format/fixed, /pgf/number format/precision=2 }
                ]

        \addplot [ybar interval, color = black, fill = dark-blue, line width = 0.5mm] 
            table {../data/hist1.data};
    \end{axis}
\end{tikzpicture}
\caption{Гистограмма $\left \{ x_i \right\}$}
\end{figure}

Оценим $M\xi$ средним арифметическим: 

\begin{equation}
\bar x = \frac{1}{n}\sum \limits_{i=1}^{n}x_i \approx 6.08
\end{equation}

Дисперсию $D\xi$ будем оценивать формулой:

\begin{equation}
s_x^2 = \frac{1}{n - 1} \sum \limits_{i = 1}^{n} x_i^2 - \frac{n}{n - 1} \bar x^2 \approx 8.2562
\end{equation}

Оценка среднеквадратического отклонения: 

\begin{equation}
s_x = \sqrt{s_x^2} \approx 2.8734
\end{equation}

\newpage

\subsubsection*{Интервальные оценки числовых характеристик}

Так как случайная величина
\begin{equation} \label{eq:1}
\frac{s^2}{\sigma^2} = \frac{1}{n-1} \sum \limits_{i = 1}^{n} \frac{\left( \xi_i - \bar \xi \right)^2}{\sigma^2} \in \frac{1}{n - 1}\chi_{n-1}^2,
\end{equation}

тогда получим, что

\begin{equation} \label{eq:2}
\frac{\bar \xi - m}{s}\sqrt{n} = 
\frac{\frac{\bar \xi - m}{\sigma}\sqrt{n}}{\frac{s}{\sigma}} = 
\frac{\frac{\bar \xi - m}{\sigma}\sqrt{n}}{\sqrt{\frac{s^2}{\sigma^2}}} \in t_{n-1}.
\end{equation}

Выберем уровень доверия $\gamma = 0.95$. Тогда для выбранного уровня доверия определим величину $t_\gamma$ такую, что будет выполняться $P\left( |t_{n-1}| < t_\gamma \right) = \gamma$. Воспользуемся функцией, обратной функции распределения, \lstinline{tinv} из пакета Matlab.

\begin{lstlisting}
tinv(       (1 - 0.95)/2, 99) = -1.98;
tinv(0.95 + (1 - 0.95)/2, 99) = 1.98;
\end{lstlisting}

% \begin{figure}[h] \label{fig:1}
% \centering
% \begin{tikzpicture}[scale = .6]
%     \begin{axis}[ height = 0.4\paperheight
%                 , width = 0.7\paperwidth
%                 , axis x line = bottom
%                 , axis y line = middle
%                 , xmin = -5.5
%                 , xmax = 5.5
%                 , ymax = 0.43
%                 , yticklabel style = { /pgf/number format/fixed, /pgf/number format/precision=2 }
%                 ]

%         \addplot [color = dark-blue, line width = 0.5mm] 
%             table {../data/t99pdf.data};
%     \end{axis}

%     \draw (6, 9.5) node {\footnotesize{$y$}};
%     \draw (13.5, 0.4) node {\footnotesize{$x$}};

%     \draw (8, 4) arc [start angle = -90, end angle = -30, radius = 1cm] 
%                  node [near end, above, minimum height = 25] {\large{$\gamma$}};
%     \draw (9.5, .5) arc [start angle = 150, end angle = 130, radius = 1.5cm] 
%                  node [near end, above] {\large{$\frac{1-\gamma}{2}$}};
%     \draw (4, .5) arc  [start angle = 30, end angle = 50, radius = 1.5cm] 
%                  node [near end, above] {\large{$\frac{1-\gamma}{2}$}};

%     \draw (4.44, -.5) -- ++(0, 3) [dashed, color = dark-green, line width = 1];
%     \draw (9.1, -.5) -- ++(0, 3) [dashed, color = dark-green, line width = 1];
%     \draw (4.44, -1) node {\large{$-t_{\gamma}$}};
%     \draw (9.1, -1)   node {\large{$t_{\gamma}$}};        
% \end{tikzpicture}
% \caption{Схема выделения множества, на котором $P\left(|t_{n-1}| < t_\gamma \right) = \gamma$}
% \end{figure}

Преобразуем левую часть уравнения (\ref{eq:2}):

\begin{equation*}
\begin{split}
\bar x - \frac{s_x}{\sqrt{n}}t_\gamma & < M\xi < \bar x + \frac{s_x}{\sqrt{n}}t_\gamma \\
\frac{s_x}{\sqrt{n}}t_\gamma & = \frac{2.8734}{10} \cdot 1.98 = 0.5689 \\
             \bar x - 0.5689 & < M\xi < \bar x + 0.5689 \\
             5.5111          & < M\xi < 6.6489
\end{split}
\end{equation*}

Следовательно, $M\xi \in \left ( 5.5111, 6.6489 \right )$ с уровнем доверия $\gamma = 0.95$.

Теперь будем искать доверительный интервал для $D\xi$. Рассмотрим случайную величину $U = (n-1)\frac{s^2}{\sigma_\xi^2}$. Из~уравнения (\ref{eq:1}) видно, что $U \in \chi^2_{n-1}$. Выберем интервал $(U_1, U_2)$ так, чтобы вероятность попадания $U$ левее и правее него была одинаковой и равной, следовательно, $(1 - \gamma)/2$. При доверительной вероятности $\gamma = 0.95$ имеем $(1 - 0.95)/2 = 0.025$. Для вычисления значения $U_1$ и $U_2$ воспользуемся функцией \lstinline{chi2inv} из пакета Matlab. 

\begin{lstlisting}
chi2inv(    0.025, 99) = 73.36
chi2inv(1 - 0.025, 99) = 128.42
\end{lstlisting}

% \begin{figure}[h] \label{fig:1}
% \centering
% \begin{tikzpicture}[scale = .6]
%     \begin{axis}[ height = 0.4\paperheight
%                 , width = 0.7\paperwidth
%                 , axis x line = bottom
%                 , axis y line = middle
%                 , xmin = 35
%                 , xmax = 165
%                 , ymax = 0.033
%                 , yticklabel style = { /pgf/number format/fixed, /pgf/number format/precision=2 }
%                 ]

%         \addplot [color = dark-blue, line width = 0.5mm] 
%             table {../data/chi2_99pdf.data};
%     \end{axis}

%     \draw (8, 4) arc [start angle = -90, end angle = -30, radius = 1cm] 
%                  node [near end, above, minimum height = 25] {\large{$\gamma$}};
%     \draw (3.4,  .4) arc  [start angle = 30, end angle = 50, radius = 2cm] 
%                      node [near end, above] {\large{$\frac{1-\gamma}{2}$}};
%     \draw (10.1, .4) arc [start angle = 150, end angle = 130, radius = 2cm] 
%                      node [near end, above] {\large{$\frac{1-\gamma}{2}$}};                 

%     \draw (4, -.5) -- ++(0, 3) [dashed, color = dark-green, line width = 1];
%     \draw (9.7, -.5) -- ++(0, 3) [dashed, color = dark-green, line width = 1];
%     \draw (3.53, -1) node {\large{$-t_{\gamma}$}};
%     \draw (10, -1)   node {\large{$t_{\gamma}$}};     
% \end{tikzpicture}
% \caption{Схема выделения множества, на котором $P\left(|t_{n-1}| < t_\gamma \right) = \gamma$}
% \end{figure}

Тогда получаем:

\begin{equation}
\begin{split}
73.36 & < 99 \cdot \frac{8.2562}{\sigma_\xi^2} < 128.42 \\
 6.36 & < D\xi                                 < 11.14
\end{split}
\end{equation}

Следовательно $D\xi \in \left ( 6.36, 11.14 \right )$ с уровнем доверия $\gamma = 0.95$.

\newpage

\subsubsection*{Проверка гипотезы о равенстве значения числовой характеристики фиксированному числу}

Проверим гипотезу о равенстве значения $M\xi$ некоторому фиксированному числу. Сформулируем простую основную гипотезу и сложную альтернативную:

\begin{equation*}
\begin{split}
H_0 & : M\xi = \mu_0 = [\bar x] = 6 \\
H_1 & : M\xi \ne \mu_0 = 6
\end{split}
\end{equation*}

Запишем критерий проверки справедливости гипотезы:

\begin{equation}
T = t_{n-1} = \frac{\bar x - \mu_0}{s}\sqrt{n}
\end{equation}

Выберем уровень значимости $\alpha = 0.01$, тогда поскольку критерий подчиняется закону распределения Стьюдента с параметром $n-1$, тогда определим $t_\text{кр.}$, исходя из того, что $P\left( |t_{n-1}| > t_\text{кр.} \right) = \alpha$.

% \begin{figure}[h] \label{fig:2}
% \centering
% \begin{tikzpicture}[scale = .6]
%     \begin{axis}[ height = 0.4\paperheight
%                 , width = 0.7\paperwidth
%                 , axis x line = bottom
%                 , axis y line = middle
%                 , xmin = -5.5
%                 , xmax = 5.5
%                 % , xlabel = $x$
%                 % , ylabel = $t_{n-1}$
%                 , ymax = 0.43
%                 , yticklabel style = { /pgf/number format/fixed, /pgf/number format/precision=2 }
%                 ]

%         \addplot [color = dark-blue, line width = 0.5mm] 
%             table {../data/t99pdf.data};
%     \end{axis}

%     \draw (6.5, 9.5) node {\footnotesize{$y$}};
%     \draw (13.5, 0.4) node {\footnotesize{$x$}};

%     \draw (8, 4) arc [start angle = -90, end angle = -30, radius = 1cm] 
%                  node [near end, above, minimum height = 25] {\footnotesize{$1 - \alpha$}};
%     \draw (10.1, .1) arc [start angle = 150, end angle = 130, radius = 1.5cm] 
%                  node [near end, above] {{$\frac{\alpha}{2}$}};
%     \draw (3.4, .1) arc  [start angle = 30, end angle = 50, radius = 1.5cm] 
%                  node [near end, above] {{$\frac{\alpha}{2}$}};

%     \draw (3.53, -1) -- ++(0, 2) [dashed, color = dark-green, line width = 1];
%     \draw (10,   -1) -- ++(0, 2) [dashed, color = dark-green, line width = 1];
%     \draw (3.53, -1.5) node {\large{$-t_{\text{кр.}}$}};
%     \draw (10,   -1.5) node {\large{$ t_{\text{кр.}}$}};
%     \draw [decorate, decoration = {brace, amplitude=10pt, mirror}] 
%           (0, -.5) -- (3.53, -.5) 
%           node [midway, yshift = -.5cm] {$S_{\text{крит.}}$};
%     \draw [decorate, decoration = {brace, amplitude=10pt, mirror}] 
%           (3.54, -.5) -- (10, -.5)
%           node [midway, yshift = -.5cm] {$Q_{\text{доп.}}$};
%     \draw [decorate, decoration = {brace, amplitude=10pt, mirror}] 
%           (10, -.5) -- ++(3.53, 0)
%           node [midway, yshift = -.5cm] {$S_{\text{крит.}}$};          
% \end{tikzpicture}
% \caption{Схема выделения множеств $Q_{\text{доп.}}$ и $S_{\text{крит.}}$}
% \end{figure}

Используя функцию, обратную функции распределения \lstinline{tinv}, получим:

\begin{lstlisting}
tinv(             0.01/2, 99) = -2.62
tinv((1 - 0.01) + 0.01/2, 99) =  2.62
\end{lstlisting}

Вычислим $T_\text{набл.}$:

\begin{equation}
T_\text{набл.} = \frac{\bar x - \mu_0}{s_x} \cdot \sqrt{n} = \frac{6.08 - 6}{2.8734} \cdot 10 = 0.27842
\end{equation}

Так как $T_\text{набл.} \in Q_\text{доп.} \Rightarrow$ оснований отвергать гипотезу $H_0$ у нас нет, и мы её принимаем.
Вычислим вероятности ошибок второго рода полагая: $H'_1 : \mu_1 = 5.5$ и $H''_2 : \mu_2 = 6.5$. Согласно теореме Леви, будем считать, что распределение вероятностей $t_{n-1}$ мало отличается от $N\left(\bar y - \mu_1; \frac{s_x}{\sqrt{n}} \right)$. тогда вероятность ошибки второго рода будет:

\begin{equation*}
\beta = \Phi \left( \frac{\mu_0 - \mu_1}{s}\sqrt{n} + t_{\text{кр.}}\right) - 
        \Phi \left( \frac{\mu_0 - \mu_1}{s}\sqrt{n} - t_{\text{кр.}}\right).
\end{equation*}

Получим $\beta' = \beta'' = 0.81054$.

Проверим гипотезу равенства дисперсии определённому числу. 
Запишем гипотезы, критерий и уровень значимости:

\begin{equation*}
\begin{split}
&H_0 : D\xi = [s^2_x] = 8 \\
&H_1 : D\xi \ne 8 \\ 
&T = \chi^2_{n-1} = \left(n-1\right)\frac{s^2_x}{\sigma^2_0} \\
&\alpha = 0.01
\end{split}
\end{equation*}

С помощью \lstinline{chi2inv} вычислим $\chi^2_{\text{крит.}}$, такое что $P\left( \chi^2_{n-1} > \chi^2_{\text{крит.}} \right) = \alpha$. 

\begin{lstlisting}
chi2inv(1 - 0.01, 99) = 134.64
\end{lstlisting}

% \begin{figure}[h] \label{fig:2}
% \centering
% \begin{tikzpicture}[scale = .6]
%     \begin{axis}[ height = 0.4\paperheight
%                 , width = 0.7\paperwidth
%                 , axis x line = bottom
%                 , axis y line = middle
%                 , xmin = 35
%                 , xmax = 165
%                 , ymax = 0.033
%                 , yticklabel style = { /pgf/number format/fixed, /pgf/number format/precision=3 }
%                 ]

%         \addplot [color = dark-blue, line width = 0.5mm] 
%             table {../data/chi2_99pdf.data};
%     \end{axis}

%     \draw (6, 5) arc  [start angle = 260, end angle = 220, radius = 3]
%                  node [near end, above, minimum height = 30] {$1 - \alpha$};
%     \draw (10.7, 0.1) arc [start angle = -50, end angle = -20, radius = 2]
%                  node [near end, above, minimum height = 20] {$\alpha$};


%     \draw (1, 9.5) node {\footnotesize{$y$}};
%     \draw (13.5, 0.4) node {\footnotesize{$x$}};
%     \draw (10.4,   -1) -- ++(0, 2) [dashed, color = dark-green, line width = 1];
%     \draw (10.6,   -1.5) node {\large{$ \chi^2_{\text{кр.}}$}};
%     \draw [decorate, decoration = {brace, amplitude=10pt, mirror}] 
%           (0, -.5) -- (10.4, -.5)
%           node [midway, yshift = -.5cm] {$Q_{\text{доп.}}$};
%     \draw [decorate, decoration = {brace, amplitude=10pt, mirror}] 
%           (10.4, -.5) -- ++(3.53, 0)
%           node [midway, yshift = -.5cm] {$S_{\text{крит.}}$};          
% \end{tikzpicture}
% \caption{Схема выделения множеств $Q_{\text{доп.}}$ и $S_{\text{крит.}}$}
% \end{figure}

Вычислим $T_{\text{набл.}}$:

\begin{equation}
T_{\text{набл.}} = \left(n - 1\right)\frac{s^2_x}{\sigma^2_0} = 99 \cdot \frac{8.2562}{8} = 102.17
\end{equation}

Поскольку $T_{\text{набл.}} \in Q_{\text{доп.}} \Rightarrow$ оснований не принимать гипотезу $H_0$ нет, мы её принимаем.

\newpage

\subsubsection*{Проверка гипотезы о совпадении значений одноимённых числовых характеристик двух различных случайных величин}

Проверим гипотезу о совпадении дисперсий двух случайных величин.
Рассмотрим дисперсии рассматриваемой нами случайной величины $D\xi$ и другой $D\xi'$. Полученные точечные оценки дисперсий: $s_x^2 = 8.2562$ и $s_x'^2 = 9.4016$. Сформулируем гипотезы:

\begin{equation*}
\begin{split}
& H_0 : \frac{D\xi'}{D\xi} = 1 \\
& H_1 : \frac{D\xi'}{D\xi} > 1 \\
& T = f_{n'-1, n-1} = \frac{\chi^2_{n'-1} / (n' - 1)}{\chi^2_{n - 1} / (n - 1)}
\end{split}
\end{equation*}

Критерием проверки справедливости гипотезы будет отношение $T \in F(n' - 1, n - 1)$ - подчиняющееся распределению Фишера-Снедекора (F-распределению).

Выберем уровень значимости $\alpha = 0.01$, тогда:

\begin{lstlisting}
finv(1 - 0.01, 99, 99) = 1.6015
\end{lstlisting}

Вычислим $f_{\text{набл.}}$:

\begin{equation}
f_{\text{набл.}} = \frac{s'^2_x}{s^2_x} = \frac{9.4016}{8.2562}= 1.1387
\end{equation}

Поскольку $1.1387 < 1.6015 \Rightarrow T \in Q_{\text{доп.}}$, а значит мы принимаем гипотезу $H_0 : D\xi = D\xi'$.
Используем этот результат в проверке гипотезы о равенстве математических ожиданий.
Запишем гипотезы и критерий:

\begin{equation*}
\begin{split}
& H_0 : M\xi = M\xi' \\
& H_1 : M\xi \ne M\xi' \\
& T = t_{n_1 + n_2 - 2} = \frac{\bar x - \bar x'}{s} \cdot \sqrt{\frac{n_1 \cdot n_2}{n_1 + n_2}},\\
& s^2 = \frac{1}{n_1 + n_2 -2}\left( \left( n_1 - 1 \right) s^2_x + \left( n_2 - 1) s_{x'}^2\right)\right)
\end{split}
\end{equation*}

\begin{lstlisting}
tinv(1 - 0.01, 100 + 100 - 2) = 2.345
\end{lstlisting}

Следовательно, при уровне значимости $\alpha = 0.01$, $|t_{\text{крит.}}| = 2.345$. Вычислим $T_{\text{набл.}}$:

\begin{equation*}
\begin{split}
& s^2 = \frac{1}{198} \left( 99 \cdot 8.2562 + 99 \cdot 9.4016 \right) = 8.8289 \\
& T_{\text{набл.}} = \frac{6.08 - 4.82}{\sqrt{8.8289}} \cdot \sqrt{\frac{100 \cdot 100}{2 \cdot 100}}
  = 2.9985
\end{split}
\end{equation*}

Поскольку $T_{\text{набл.}} > t_{\text{крит.}} \Rightarrow$ отклоняем гипотезу $H_0$ и принимаем $H_1$, при этом вероятность ошибки первого рода равна $\alpha = 0.01$.

\newpage

\subsubsection*{Проверка гипотезы о виде закона распределения}

В нашем случае гистограмма похожа на плотность вероятности гамма-распределения. Плотность вероянтости гамма-распределения:

\begin{equation*}
p(x) = 
\begin {cases}
0, & x \le 0 \\
\frac{1}{\beta^\alpha \Gamma\left(\alpha\right)} \cdot x^{\alpha - 1} e^{-\left(\frac{x}{\beta}\right)}, & x > 0 \\
\end {cases},
\end{equation*}

где $\alpha$ и $\beta$ -- числовые параметры распределения. Найдём эти параметры:

\begin{equation*}
\begin{split}
& M\xi = \alpha \cdot \beta, D\xi = \alpha \cdot \beta^2 \Rightarrow \\
& \Rightarrow \alpha = \frac{M^2\xi}{D\xi}, \beta = \frac{D\xi}{M\xi} \\
& M\xi = 6, D\xi = 8 \Rightarrow \alpha = \frac{36}{8} = 4.5, \beta = \frac{8}{6} = 1.33
\end{split}
\end{equation*}

Критерием будет случайная величина:

\begin{equation*}
T = \chi^2_{k - l - 1} = \sum \limits_{j=1}^k \frac{\left(m_j - np_j\right)^2}{np_j}
\end{equation*}

где $k = 6$ -- количество интервалов вариационного ряда, а $l = 2$ -- количество параметров распределения, которые заменяются их точечными оценками, тогда при урове значимости $\alpha = 0.01$,  $\chi^2_{\text{крит.}} = 11.345$.

\begin{lstlisting}
chi2inv(1 - 0.01, 6 - 2 - 1) = 11.345
\end{lstlisting}

\begin{table}[h]
\begin{tabularx}{\textwidth}{|c|c|c|c|c|c|}
\hline
    & $( a_{j_1}; a_j]$ & $m_j$ & $p_j$      & $np_j$   & $(m_j - np_j)^2 / np_j$ \\
\hline 
    $1$ & $[1;3)$       & $6$   & $0.12245$  & $12.245$ & $2.6187$ \\
    $2$ & $[3;5)$       & $23$  & $0.2914 $  & $29.14$  & $1.2937$ \\
    $3$ & $[5;7)$       & $35$  & $0.27369$  & $27.369$ & $2.1277$ \\
    $4$ & $[7;9)$       & $20$  & $0.16968$  & $12.968$ & $3.8132$ \\
    $5$ & $[9;11)$      & $10$  & $0.083499$ & $8.3499$ & $0.32609$ \\
    $6$ & $[11;21]$     & $6$   & $0.056166$ & $5.6166$ & $0.026172$ \\
$\Sigma$&               & $100$ & $0.99689$  &          & $10.206$ \\
\hline
\end{tabularx}
\end{table}

Получили $\chi^2_{\text{набл.}} = 10.206 < 11.345 \Rightarrow$ у нас нет оснований отклонять гипотезу $H_0 : \xi \in \Gamma\left(\alpha, \beta\right)$.

\newpage

\subsubsection*{Проверка гипотезы о совпадении законов распределения двух случайных величин}

Имеем вариационный ряд, построенный по выборке ${x'_i}$ значений случайной величины $\xi'$ и подогнанный к разбиению вариационного ряда $\xi$, чтобы $a_i = a'_i$.

\begin{table}[h]
\begin{tabularx}{\textwidth}{|c|c|c|c|c|c|c|c|c|c|c|}
\hline
    $\left [a_i, a_{i+1}\right)$ & $\left [1, 3\right )$ & $\left [3, 5\right )$ & $\left [5, 7\right )$ & $\left [7, 9\right )$ & $\left [9, 11\right )$ & $\left [11, 13\right )$ & $\left [13, 15\right )$ & $\left [15, 17\right )$ & $\left [17, 18\right )$ & $\left [19, 21\right ]$ \\
\hline
    $\frac{m_i}{n}$ & $0.17$ & $0.35$ & $0.31$ & $0.11$ & $0.02$ & $0.01$ & $0.0$ & $0.01$ & $0.01$ & $0.01$ \\
\hline
\end{tabularx}
\setcounter{table}{1}
\caption {Вариационный ряд $\{ x'_i \}$}
\end{table}

По этой выборке так же имеем: $\bar x' = 4.82, s'^2 = 9.4016, s' = 3.0662$. Формулируем гипотезу и критерий:

\begin{equation*}
\begin{split}
& H_0 : F_{\xi}(x) = F_{\xi'}(x) \\
& T = \chi^2_{k-1} = n_1 \cdot n_2 \sum \limits_{j=1}^k \frac{1}{m_j + m'_j} \left( \frac{m_j}{n_1} - \frac{m'_j}{n_2} \right)^2 \stackrel{(n_1 = n_2 = 100)}{=} \sum \limits_{j=1}^k \frac{\left( m_j - m'_j\right)^2}{m_j + m'_j}.
\end{split}
\end{equation*}

Выбираем уровень значимости $\alpha = 0.01$, тогда получаем $\chi^2_{\text{крит.}} = 15.086$. Вычислим $\chi^2_{\text{набл.}}$:

\begin{equation*}
\chi^2_{\text{набл.}} = 16.332 
\end{equation*}

Так как $16.332 > 15.086  \Rightarrow T_{\text{набл.}} \in S_{\text{крит.}}$, значит мы отклоняем гипотезу $H_0$ с уровнем значимости $\alpha$ и принимаем гипотезу $H_1$. При этом вероятность ошибки первого рода $\alpha = 0.01$.

\newpage

\subsubsection{Первичная обработка статистических данных}

Вторая выборка $\{ y_i \}$ объёма $n = 100$.
Размах выборки: $\min(y_i) = 1.8$ и $\max(y_i) = 68.9$. 
Интервал $[1.8, 68.9]$ так же делим на $k_2 = 10$ частей. 

\begin{table}[h]
\begin{tabularx}{\textwidth}{|c|c|c|c|c|c|c|c|c|c|c|}
\hline
    $\left [a_i, a_{i+1}\right)$ & $\left [1.8, 8.51\right )$ & $\left [8.51, 15.22\right )$ & $\left [15.22, 21.93\right )$ & $\left [21.93, 28.64\right )$ & $\left [28.64, 35.35\right )$ \\
\hline
    $\frac{m_i}{n}$ & $0.08$ & $0.08$ & $0.12$ & $0.13$ & $0.14$ \\
\hline
\hline
    & $\left [35.35, 42.06\right )$ & $\left [42.06, 48.77\right )$ & $\left [48.77, 55.48\right )$ & $\left [55.48, 62.19\right )$ & $\left [62.19, 68.9\right ]$ \\
\hline
    & $0.08$ & $0.18$ & $0.04$ & $0.07$ & $0.08$ \\
\hline
\end{tabularx}
\setcounter{table}{2}
\caption {Вариационный ряд $\{ y_i \}$}
\end{table}

\begin{figure}[h] \label{fig:2}
\centering
\begin{tikzpicture} [scale = 1]
    \begin{axis}[ height = 0.4\paperheight
                , width = 0.7\paperwidth
                , xtick = data
                , ytick = data
                , yticklabel style = { /pgf/number format/fixed, /pgf/number format/precision=2 }
                ]

        \addplot [ybar interval, color = black, fill = dark-blue, line width = 0.5mm] 
            table {../data/hist2.data};

    \end{axis}
\end{tikzpicture}
\caption{Гистограмма $\left \{ y_i \right\}$}
\end{figure}

Оценка мат. ожидания $M\eta$: 

\begin{equation}
\bar y = \frac{1}{n}\sum \limits_{i=1}^{n}y_i \approx 33.923
\end{equation}

Оценка дисперсии $D\eta$:

\begin{equation}
s_y^2 = \frac{1}{n - 1} \sum \limits_{i = 1}^{n} y_i^2 - \frac{n}{n - 1} \bar y^2 \approx 323.4
\end{equation}

Оценка среднеквадратического отклонения: 

\begin{equation}
s_y = \sqrt{s_y^2} \approx 17.983
\end{equation}

\newpage

\subsubsection*{Интервальные оценки числовых характеристик}

Найдём доверительный интервал для $M\eta$. Для выбранного уровня доверия $\gamma = 0.95$, $t_\gamma = 1.98$. Тогда:

\begin{equation*}
\begin{split}
\bar y - \frac{s_y}{\sqrt{n}}t_\gamma < M\eta < \bar y + \frac{s_y}{\sqrt{n}}t_\gamma \\
\frac{s_y}{\sqrt{n}}t_\gamma & = \frac{17.983}{10} \cdot 1.98 = 3.56 \\
\bar y - 3.56   & < M\eta < \bar y + 3.56 \\
30.363          & < M\eta < 37.483
\end{split}
\end{equation*}

Следовательно, $M\eta \in \left ( 30.363, 37.483 \right )$ с уровнем доверия $\gamma = 0.95$.

Теперь будем искать доверительный интервал для $D\eta$:

\begin{equation*}
\begin{split}
\chi^2_1 & < \left(n-1\right) \cdot \frac{s^2_y}{\sigma^2_\eta} < \chi^2_2 \\
73.36  & < 99 \cdot \frac{323.4}{\sigma_\eta^2} < 128.42 \\
249.31 & < D\eta                            < 436.43
\end{split}
\end{equation*}

Следовательно $D\eta \in \left ( 249.31, 436.43 \right )$ с уровнем доверия $\gamma = 0.95$.

\subsubsection*{Проверка гипотезы о равенстве значения числовой характеристики фиксированному числу}

\begin{equation*}
\begin{split}
& H_0 : M\eta = \mu_0 = [\bar y] = 34 \\
& H_1 : M\eta \ne \mu_1 = 34 \\
& \alpha = 0.01 \\
& T = \frac{\bar y - \mu_0}{s} \sqrt{n}
\end{split}
\end{equation*}

Найдём $t_{\text{крит.}}$, исходя из того, что $P\left( |t_{n-1}| > t_{\text{кр.}} \right) = \alpha$:

\begin{lstlisting}
tinv(             0.01/2, 99) = -2.62
tinv((1 - 0.01) + 0.01/2, 99) =  2.62
\end{lstlisting}

Вычислим $T_\text{набл.}$:

\begin{equation}
T_\text{набл.} = \frac{33.923 - 34}{17.983} \cdot 10 = -0.042818
\end{equation}

Так как $0.042818 < 2.62 \Rightarrow T_\text{набл.} \in Q_\text{доп.} \Rightarrow$ оснований отвергать гипотезу $H_0$ нет, мы её принимаем. 

Теперь проверим гипотезу о равенстве $D\eta$:

\begin{equation*}
\begin{split}
& H_0 : D\eta = [s_y^2] = 323 \\
& H_1 : D\eta \ne = 323 \\
& \alpha = 0.01 \\
& T = \left(n-1\right)\frac{s^2_y}{\sigma_0^2}
\end{split}
\end{equation*}

Найдём $\chi^2_\text{крит.}$, такое что $P\left(\chi^2_{n-1} > \chi^2_\text{крит.}\right) = \alpha$.

\begin{lstlisting}
chi2inv(1 - 0.01, 99) = 134.64
\end{lstlisting}

Вычислим $T_\text{набл.}$:

\begin{equation}
T_\text{набл.} = \left(n-1\right) \frac{s_y^2}{\sigma^2_0} = 99 \cdot \frac{323.4}{323} = 99.123
\end{equation}

Поскольку $T_\text{набл.} \in Q_\text{доп.} \Rightarrow$ оснований не принимать гипотезу $H_0$ нет, мы её принимаем.

\newpage

\subsubsection{Проверка гипотезы о совпадении значений одноимённых числовых характеристик двух различных случайных величин}

Проверим гипотезу о совпадении дисперсий рассматриваемой нами случайной величины $D\eta$ и $D\eta'$. Полученные точечные оценки дисперсий: $s^2_y = 323.4$ и $s'^2_y = 815.59$. Запишем гипотезы, критерий и уровень значимости:

\begin{equation*}
\begin{split}
& H_0 : \frac{D\eta'}{D\eta} = 1 \\
& H_1 : \frac{D\eta'}{D\eta} > 1 \\
& \alpha = 0.01 \\
& T = f_{n'-1, n-1} = \frac{\chi^2_{n'-1} / \left(n' - 1\right)}{\chi^2_{n-1} / \left(n - 1\right)}
\end{split}
\end{equation*}

Для уровня значимости $\alpha = 0.01$:

\begin{lstlisting}
finv(1 - 0.01, 99, 99) = 1.6015
\end{lstlisting}

Вычислим $T_\text{набл.}$:

\begin{equation}
T_\text{набл.} = \frac{s'^2_y}{s^2_y} = \frac{815.59}{323.4} = 2.5219
\end{equation}

Поскольку $2.5219 > 1.6015 \Rightarrow T \in S_\text{крит.}$, а значит мы отклоняем гипотезу $H_0$ и принимаем $H_1$. При этом вероятность ошибки первого рода равна $\alpha = 0.01$.

Запишем гипотезы:

\begin{equation*}
\begin{split}
& H_0 : M\eta = M\eta' \\
& H_1 : M\eta \ne M\eta'
\end{split}
\end{equation*}

Поскольку гипотеза о равенстве дисперсий не подтвердилась, будем использовать другой критерий:

\begin{equation}
T = u = \left(\bar y - \bar y' \right) / \sqrt{\frac{s^2_y}{n} + \frac{s'^2_y}{n'}}
\end{equation}

Этот критерий, согласно теореме Леви, асимптотически нормален, тогда $P\left(|u| > u_\text{кр.}\right) = 1 - 2\Phi\left(u_\text{кр.}\right) = \alpha$, или $u_\text{кр.} = \Phi^{-1}\left(\left(1 - \alpha\right) / 2\right)$.

\begin{lstlisting}
norminv(    0.01/2) = -2.5758
norminv(1 - 0.01/2) =  2.5758
\end{lstlisting}

Вычислим $T_\text{набл.}$:

\begin{equation}
T_\text{набл.} = (33.923 - 51.184) / \sqrt{\frac{323.4}{100} + \frac{815.59}{100}} = -17.261 \cdot \sqrt{11.390} = -58.254
\end{equation}

Поскольку $|T_\text{набл.}| > u_\text{кр.} \Rightarrow$ мы отклоняем гипотезу $H_0$ и принимаем $H_1$.

\newpage

\subsubsection*{Проверка гипотезы о виде закона распределения}

В нашем случае гистограмма похожа на плотность вероятности равномерного распределения. Плотность вероятности равномерного распределения:

\begin{equation*}
p(x) = 
\begin {cases}
0,               & x < a \\
\frac{1}{b - a}, & a \le x < b \\
0,               & b \le x \\
\end {cases},
\end{equation*}

где $a$ и $b$ -- начало и конец отрезка, на котором $\eta$ принимает не нулевые значения. Возьмём в качестве этих параметров $x_{min} = 1.8$ и $x_{max} = 68.9$. Критерий:

\begin{equation}
T = \chi^2_{k-l-1} = \sum \limits_{j=1}^{k} \frac{\left(m_j - np_j\right)^2}{np_j}
\end{equation}

где $k = 10$ -- количество интервалов вариационного ряда, а $l = 2$ -- количество параметров распределения, которые заменяются их точечными оценками. Тогда при уровне значимости $\alpha = 0.01$, $\chi^2_\text{крит.} = 18.475$.

\begin{lstlisting}
chi2inv(1 - 0.01, 10 - 2 - 1) = 18.475
\end{lstlisting}

\begin{table}[h]
\begin{tabularx}{\textwidth}{|c|c|c|c|c|c|}
\hline
      & $( a_{j_1}; a_j]$ & $m_j$ & $p_j$  & $np_j$ & $(m_j - np_j)^2 / np_j$ \\
\hline 
    $1$ & $[1.8  ;8.51 )$ & $8$   & $0.1$  & $10$   & $0.39996$ \\
    $2$ & $[8.51 ;15.22)$ & $8$   & $0.1$  & $10$   & $0.39996$ \\
    $3$ & $[15.22;21.93)$ & $12$  & $0.1$  & $10$   & $0.40004$ \\
    $4$ & $[21.93;28.64)$ & $13$  & $0.1$  & $10$   & $0.90007$ \\
    $5$ & $[28.64;35.35)$ & $14$  & $0.1$  & $10$   & $1.6001 $ \\
    $6$ & $[35.35;42.06)$ & $8$   & $0.1$  & $10$   & $0.39996$ \\
    $7$ & $[42.06;48.77)$ & $18$  & $0.1$  & $10$   & $6.4002$ \\
    $8$ & $[48.77;55.48)$ & $4$   & $0.1$  & $10$   & $3.5999$ \\
    $9$ & $[55.48;62.19)$ & $7$   & $0.1$  & $10$   & $0.89995$ \\
    $10$& $[62.19;68.9 ]$ & $8$   & $0.1$  & $10$   & $0.39996$ \\
$\Sigma$&                 & $100$ & $1$    &        & $15.400$ \\
\hline
\end{tabularx}
\end{table}

Получили $\chi^2_\text{набл.} = 15.4 < 18.475 \Rightarrow$ у нас нет оснований отклонять гипотезу $H_0$.

\newpage

\subsubsection*{Проверка гипотезы о совпадении законов распределения двух случайных величин}

Перепишем оба вариационных ряда так, чтобы разбиения у них совадали ($a_i = a'_i$).

\begin{table}[h]
\begin{tabularx}{\textwidth}{|c|c|c|c|c|c|c|c|c|c|c|}
\hline
    $\left [a_i, a_{i+1}\right)$ & $[0, 10)$ & $[10,20)$ & $[20,30)$ & $[30,40)$ & $[40,50)$ \\
\hline
    $\frac{m_i}{n}$ & $8$ & $19$ & $19$ & $14$ & $21$ \\
\hline
\hline
    & $[50,60)$ & $[60,70)$ & $[70,80)$ & $[80,90)$ & $[90,100]$ \\
\hline
    & $9$ & $10$ & $0$ & $0$ & $0$ \\
\hline
\end{tabularx}
\setcounter{table}{2}
\caption {Вариационный ряд $\{ y_i \}$}
\end{table}

\begin{table}[h]
\begin{tabularx}{\textwidth}{|c|c|c|c|c|c|c|c|c|c|c|}
\hline
    $\left [a_i, a_{i+1}\right)$ & $[0, 10)$ & $[10,20)$ & $[20,30)$ & $[30,40)$ & $[40,50)$ \\
\hline
    $\frac{m_i}{n}$ & $11$ & $8$ & $8$ & $10$ & $13$ \\
\hline
\hline
    & $[50,60)$ & $[60,70)$ & $[70,80)$ & $[80,90)$ & $[90,100]$ \\
\hline
    & $8$ & $14$ & $7$ & $8$ & $13$ \\
\hline
\end{tabularx}
\setcounter{table}{2}
\caption {Вариационный ряд $\{ y'_i \}$}
\end{table}

Мы видим, что в первом вариационном ряду значения на отрезках $[70, 80), [80, 90)$ и $[90, 100)$ равны нулю. Мы можем сделать проверку двумя способами: 1) можно выбросить часть вариационного ряда $y'_i$, и 2) объединим интервалы в $[60,100)$ в обоих вариационных рядах.

Рассмотрим вариант 1):

Имеем гипотезу: $H_0 : F_{\eta}(x) = F_{\eta'}(x)$;

Тогда, поскольку мы выбросили часть выборки $\{y'_i\}$, тогда $n_2 = 100 - 13 - 8 - 8 = 71$ и критерий будет иметь вид:

\begin{equation}
T = \chi^2_{k-1} = n_1 \cdot n_2 \sum \limits^{k}_{j=1} \frac{1}{m_j + m'_j} \left(\frac{m_j}{n_1} - \frac{m'_j}{n_2}\right)^2
\end{equation}

Для уровня значимости $\alpha = 0.01$ получим $\chi^2_\text{крит.} = 16.812$

Вычислим $T_\text{набл.} = 8.3750 < 16.812 \Rightarrow H_0$ принимается.

Рассмотрим вариант 2):

Гипотеза $H_0$ - та же. Поскольку $n_1 = n_2 = 100$, то воспользуемся более простым критерием:

\begin{equation}
T = \sum \limits^{k}_{j = 1} \frac{\left(m_j - m'_j\right)^2}{m_j + m'_j}.
\end{equation}

Тогда при тех же $\alpha = 0.01$ и $\chi^2_\text{крит.} = 16.812$ получим $\chi^2_\text{набл.} = 31.737 > 16.812 \Rightarrow H_0$ отклоняется.

Какой из способов выбрать? Мы знаем, что оба распределения - равномерны, но их графики, видимо, выглядят так:

\begin{figure}[h] \label{fig:3}
\centering
\begin{tikzpicture}[scale = .8]
    \begin{axis}[ height = 0.2\paperheight
                , width = 0.7\paperwidth
                , xmin = 35
                , xmax = 165
                , ymax = 0.033
                , axis x line = bottom
                , axis y line = middle
                ]

    \end{axis}

    \draw [color = dark-green, line width = 1]
        (0, 0) -- ++(0, 1) -- ++(13.2,0) -- ++(0, -1);
    \draw [color = dark-blue, line width = 1]
        (1.5, 0) -- ++(0, 1.7) -- ++(6, 0) -- ++ (0, -1.7);
      
\end{tikzpicture}
\end{figure}

\newpage

\subsubsection*{Корреляционный анализ}

По элементам двумерной выборки $\{ (x_i;y_i) : y \in \bar {1,n} \}$ составим корреляционную таблицу:

\begin{table}[h]
\begin{tabularx}{\textwidth}{|c|c|c|c|c|c|c|c|c|c|c|c|}
\hline
  $\eta \backslash \xi$ & $[1,3)$ & $[3,5)$ & $[5,7)$ & $[7,9)$ & $[9,11)$ & $[11,13)$ & $[13,15)$ & $[15,17)$ & $[17,19)$ & $[19,21]$ & \\
\hline 
    $[ 1.8 ; 8.51)$ &     & $1$ &      & $2$  & $2$ & $3$ &     & & &     & $8$  \\
    $[ 8.51;15.22)$ & $1$ &     & $3$  & $2$  & $1$ & $1$ &     & & &     & $8$  \\
    $[15.22;21.93)$ &     &     & $3$  & $3$  & $5$ &     & $1$ & & &     & $12$ \\
    $[21.93;28.64)$ &     &     & $6$  & $5$  & $1$ &     &     & & & $1$ & $13$ \\
    $[28.64;35.35)$ &     & $2$ & $10$ & $1$  & $1$ &     &     & & &     & $14$ \\
    $[35.35;42.06)$ &     & $3$ & $3$  & $2$  &     &     &     & & &     & $8$  \\
    $[42.06;48.77)$ &     & $4$ & $10$ & $4$  &     &     &     & & &     & $18$ \\
    $[48.77;55.48)$ & $1$ & $2$ &      & $1$  &     &     &     & & &     & $4$  \\
    $[55.48;62.19)$ & $1$ & $6$ &      &      &     &     &     & & &     & $7$  \\
    $[62.19;68.9 ]$ & $3$ & $5$ &      &      &     &     &     & & &     & $8$  \\
\hline
                   & $6$ & $23$ & $35$ & $20$ & $10$ & $4$ & $1$ & & & $1$ & $100$\\
\hline
\end{tabularx}
\end{table}

Сила статистической связи оценивается с помощью коэффициента линейной корреляции:

\begin{equation}
\rho = \frac{\alpha_{11} - m_\xi m\eta}{\sigma_\xi \sigma_\eta}
\end{equation}

Точечная оценка коэффициента линейной корреляции:

\begin{equation}
\begin{split}
& r = \frac{a_{11} - \bar x \bar y}{s_x s_y}, \text{где} \\
& a_{11} = \frac{1}{n} \sum \limits^{n}_{i=1}x_i y_i  
\end{split}
\end{equation}

В нашем случае $r = -0.011603$. Получнное близкое к нулю значение позволяет сделать вывод о почти отсутствующей связи между случайными величинами. 

Найдём функции регрессии: $f(x) = M\left[\eta / \left\{\xi = x\right\}\right]$.
По корреляционной таблице видно, что обе функции регрессии - линейные, то есть $f(x) = a_1x+b_1$. Теоретическое уравнение линейной функции регрессии $\eta$ на $\xi$ имеет вид:

\begin{equation}
y - m_\eta = \rho \frac{\sigma_\eta}{\sigma_\xi} \left( x - m_\xi \right)
\end{equation}

Заменяя в этом уравнении теоретические числовые характеристики их точечными оценками получим:

\begin{equation}
y - \bar y = r \frac{s_y}{s_x} \left( x - \bar x \right)
\end{equation}

В нашем случае уравнение принимает вид:

\begin{equation}
\begin{split}
& y - 33.923 = -0.011603 \cdot \frac{17.983}{2.8734} \left(x - 6.08\right) \\
& y - 33.923 = -0.072617 \cdot \left(x - 6.08\right) \\
& y = -0.072617 x + 34.365
\end{split}
\end{equation}

Аналогично находим регрессию $\xi$ на $\eta$:

\begin{equation}
\begin{split}
& x - m_\xi = \rho \frac{\sigma_\xi}{\sigma_\eta} \left( y - m_\eta \right) \\
& x - \bar x = r \frac{s_x}{s_y} \left( y - \bar y \right) \\
& x - 6.08 = -0.011603 \cdot \frac{2.8734}{17.983} \left(y - 33.923\right) \\
& x = -0.0018539 y + 6.1429
\end{split}
\end{equation}

\newpage

\begin{figure}[h] \label{fig:4}
\centering
\begin{tikzpicture}[scale = .8]
    \begin{axis}[ height = 0.4\paperheight
                , width = 0.7\paperwidth
                , xmin = 0
                , xmax = 5
                , ymax = 34.5
                , axis x line = bottom
                , axis y line = left
                ]
        \addplot [dark-blue] {-0.072617*x + 34.365};
    \end{axis}
\end{tikzpicture}
\end{figure}

\begin{figure}[h] \label{fig:4}
\centering
\begin{tikzpicture}[scale = .8]
    \begin{axis}[ height = 0.4\paperheight
                , width = 0.7\paperwidth
                , xmin = 0
                , xmax = 5
                , ymax = 3500
                , axis x line = bottom
                , axis y line = left
                ]
        \addplot [dark-green]{-539.40*x + 3313.5};
    \end{axis}
\end{tikzpicture}
\end{figure}

\end{document}
