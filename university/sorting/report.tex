\documentclass[12pt]{article}

\usepackage[T1]{fontenc}
\usepackage[utf8]{inputenc}
\usepackage[english, russian]{babel} 
\usepackage[scaled = 0.8]{PTMono} %monotype font
% \usepackage{paratype} %fonts
\usepackage{indentfirst} %indent after sections
\usepackage{xspace} % for additional spaces

\usepackage[left = 2cm, right = 1cm, top = 2cm, bottom = 2cm]{geometry}
\usepackage{lastpage}
\usepackage{setspace}
\onehalfspacing

\usepackage%
[pdftex,unicode,pdfborder={0 0 0},draft=false,%backref=page,
    hidelinks, % убрать, если хочется видеть ссылки: это
               % удобно в PDF файле, но не должно появиться на печати
    bookmarks=true,bookmarksnumbered=false,bookmarksopen=false]%
{hyperref}

\usepackage{tabularx}
\usepackage{ltablex}  % for centering inside cells of tables
\usepackage{pgfplots} % for histograms
\usepackage{xcolor}   % for defining new colors
\usepackage{amsmath}  % for split in equations
\usepackage{amssymb}  % for >= <=
\renewcommand {\le} {\leqslant}
\renewcommand {\ge} {\geqslant}

\definecolor{dark-blue}{rgb}{0.108, 0.3137, 0.70196}
\definecolor{dark-green}{rgb}{0.108, 0.7, 0.3}
\definecolor{good-red}{rgb}{0.7, 0.1, 0.3}

\usepackage{tikz}
\usepackage{tikz-qtree}
\usetikzlibrary{scopes, 
  positioning, 
  chains, 
  arrows.meta, 
  shadows}

\usepackage{listings}
\lstset { basicstyle = \ttfamily
        , showspaces = false
        , showstringspaces = false
        , language = C++
        }

\begin{document}

\thispagestyle{empty}
\begin{centering}

МИНИСТЕРСТВО ОБРАЗОВАНИЯ И НАУКИ РФ\\
Федеральное государственное образовательное учреждение\\ высшего образования\\
<<ЮЖНЫЙ ФЕДЕРАЛЬНЫЙ УНИВЕРСИТЕТ>> \\
\vspace{1cm}
Институт математики, механики и компьютерных наук им. И.\,И.~Воровича\\
\vspace{6cm}

{\Large \textbf{Отчёт} } \\
<<Распараллеливание сортировки>>\\
\end{centering}

\vfill

\begin{flushright}
\begin{tabular}{l}
Выполнил: cтудент III курса, {Радько~В.\,А.} \\
Проверил: доцент каф.\,ВМиМФ, {Говорухин~В.\,Н.} \\
\end{tabular}
\end{flushright}

\vspace{4cm}

\begin{centering}
Ростов-на-Дону, 2018\\
\end{centering}

\newpage

\tableofcontents

\newpage

\section{Распараллеливание сортировки с помощью OpenMP}

В качестве базового способа сортировки была выбрана простейшая сортировка <<пузырьком>>. Метод распараллеливания заключается в том, чтобы разбить массив на приблизительно равные участки, отсортировать каждый в отдельности, а затем слить эти участки в один отсортированный массив. Массив разбивается на $k$ участков $[a_i, a_{i+1}]$, где $a_i = a_0 + i\cdot\dfrac{n}{k}$, а длина такого участка будет равна $(i+1)\cdot\dfrac{n}{k} - i\cdot\dfrac{n}{k}$.

Все следующие тесты проводились $10$ раз, и бралось среднее значение, чтобы приблизить результаты эксперимента к математическому ожиданию.

\subsection{Тест $p=1$, $k$ -- переменная}

Тест при фиксированном одном потоке и переменном количестве разбиений показывает, что алгоритм сортировки <<пузырьком>> ускоряется даже на одном потоке. Результаты такого теста на рис. \ref{fig:1}

\begin{figure}[h] 
\centering
\begin{tikzpicture}[scale = 1]
    \begin{axis}[ height = 0.4\paperheight
                , width = 0.7\paperwidth
                % , axis x line = bottom
                % , axis y line = middle
                , xtick = data
                , ytick = data
                , xmin = 0
                , xmax = 380000
                % , ymax = 0.43
                , xlabel = n
                , ylabel = micro seconds
                , yticklabel style = { /pgf/number format/fixed, /pgf/number format/precision=2 }
                ]

        \addplot [color = dark-blue, line width = 0.5mm, mark = *] 
            table {data/sort p=1 k=1.data};
        \addlegendentry{$k = 1$}

        \addplot [color = dark-green, line width = 0.5mm, mark = *]
            table {data/sort p=1 k=2.data};
        \addlegendentry{$k = 2$}

        \addplot [color = orange, line width = 0.5mm, mark = *]
            table {data/sort p=1 k=4.data};
        \addlegendentry{$k = 4$}

        \addplot [color = good-red, line width = 0.5mm, mark = *]
            table {data/sort p=1 k=8.data};
        \addlegendentry{$k = 8$}
    \end{axis}
\end{tikzpicture}
\caption{\label{fig:1} Ускорение при увеличении параметра $k$, один поток}
\end{figure}

\newpage

\subsection{Тест при увеличении обоих параметров}

Результаты на рисунках \ref{fig:2}--\ref{fig:5} показывают, что при $k < p$ результатов от параллельности нет, поскольку алгоритм параллелится по участкам, которых $k$ штук.

\begin{figure}[h]
\begin{minipage}{0.45\textwidth}
    \begin{tikzpicture}[scale = 1]
        \begin{axis}[
                % , axis x line = bottom
                % , axis y line = middle
                , xtick = data
                , ytick = data
                , xmin = 0
                , xmax = 380000
                % , ymax = 0.43
                , xlabel = n
                , ylabel = micro seconds
                , yticklabel style = { /pgf/number format/fixed, /pgf/number format/precision=2 }
                ]

            \addplot [color = dark-blue, line width = 0.5mm, mark = *] 
                table {data/sort p=1 k=1.data};
            \addlegendentry{$p = 1$}

            \addplot [color = dark-green, line width = 0.5mm, mark = *] 
                table {data/sort p=1 k=1.data};
            \addlegendentry{$p = 2$}

            \addplot [color = orange, line width = 0.5mm, mark = *] 
                table {data/sort p=1 k=1.data};
            \addlegendentry{$p = 4$}

            \addplot [color = good-red, line width = 0.5mm, mark = *] 
                table {data/sort p=1 k=1.data};
            \addlegendentry{$p = 6$}

            \addplot [color = dark-blue, line width = 0.5mm, mark = *] 
                table {data/sort p=1 k=1.data};
            \addlegendentry{$p = 8$}
        \end{axis}
    \end{tikzpicture}
    \caption{\label{fig:2}Ускорение при фиксации параметра $k = 1$}
\end{minipage} \hfill
\begin{minipage}{0.45\textwidth}
    \begin{tikzpicture}[scale = 1]
        \begin{axis}[
                % , axis x line = bottom
                % , axis y line = middle
                , xtick = data
                , ytick = data
                , xmin = 0
                , xmax = 380000
                % , ymax = 0.43
                , xlabel = n
                , ylabel = micro seconds
                , yticklabel style = { /pgf/number format/fixed, /pgf/number format/precision=2 }
                ]

            \addplot [color = dark-blue, line width = 0.5mm, mark = *] 
                table {data/sort p=1 k=2.data};
            \addlegendentry{$p = 1$}

            \addplot [color = dark-green, line width = 0.5mm, mark = *] 
                table {data/sort p=2 k=2.data};
            \addlegendentry{$p = 2$}

            \addplot [color = orange, line width = 0.5mm, mark = *] 
                table {data/sort p=2 k=2.data};
            \addlegendentry{$p = 4$}

            \addplot [color = good-red, line width = 0.5mm, mark = *] 
                table {data/sort p=2 k=2.data};
            \addlegendentry{$p = 6$}

            \addplot [color = dark-blue, line width = 0.5mm, mark = *] 
                table {data/sort p=2 k=2.data};
            \addlegendentry{$p = 8$}
        \end{axis}
    \end{tikzpicture}
    \caption{\label{fig:3}Ускорение при фиксации параметра $k = 2$}
\end{minipage}
\end{figure}

\begin{figure}[h]
\begin{minipage}{0.45\textwidth}
    \begin{tikzpicture}[scale = 1]
        \begin{axis}[
                % , axis x line = bottom
                % , axis y line = middle
                , xtick = data
                , ytick = data
                , xmin = 0
                , xmax = 380000
                % , ymax = 0.43
                , xlabel = n
                , ylabel = micro seconds
                , yticklabel style = { /pgf/number format/fixed, /pgf/number format/precision=2 }
                ]

            \addplot [color = dark-blue, line width = 0.5mm, mark = *] 
                table {data/sort p=1 k=4.data};
            \addlegendentry{$p = 1$}

            \addplot [color = dark-green, line width = 0.5mm, mark = *] 
                table {data/sort p=2 k=4.data};
            \addlegendentry{$p = 2$}

            \addplot [color = orange, line width = 0.5mm, mark = *] 
                table {data/sort p=4 k=4.data};
            \addlegendentry{$p = 4$}

            \addplot [color = good-red, line width = 0.5mm, mark = *] 
                table {data/sort p=4 k=4.data};
            \addlegendentry{$p = 6$}

            \addplot [color = dark-blue, line width = 0.5mm, mark = *] 
                table {data/sort p=4 k=4.data};
            \addlegendentry{$p = 8$}
        \end{axis}
    \end{tikzpicture}
    \caption{\label{fig:4}Ускорение при фиксации параметра $k = 6$}
\end{minipage} \hfill
\begin{minipage}{0.45\textwidth}
    \begin{tikzpicture}[scale = 1]
        \begin{axis}[
                % , axis x line = bottom
                % , axis y line = middle
                , xtick = data
                , ytick = data
                , xmin = 0
                , xmax = 380000
                % , ymax = 0.43
                , xlabel = n
                , ylabel = micro seconds
                , yticklabel style = { /pgf/number format/fixed, /pgf/number format/precision=2 }
                ]

            \addplot [color = dark-blue, line width = 0.5mm, mark = *] 
                table {data/sort p=1 k=8.data};
            \addlegendentry{$p = 1$}

            \addplot [color = dark-green, line width = 0.5mm, mark = *] 
                table {data/sort p=2 k=8.data};
            \addlegendentry{$p = 2$}

            \addplot [color = orange, line width = 0.5mm, mark = *] 
                table {data/sort p=4 k=8.data};
            \addlegendentry{$p = 4$}

            \addplot [color = good-red, line width = 0.5mm, mark = *] 
                table {data/sort p=6 k=8.data};
            \addlegendentry{$p = 6$}

            \addplot [color = dark-blue, line width = 0.5mm, mark = *] 
                table {data/sort p=8 k=8.data};
            \addlegendentry{$p = 8$}
        \end{axis}
    \end{tikzpicture}
    \caption{\label{fig:5}Ускорение при фиксации параметра $k = 8$}
\end{minipage}
\end{figure}

\newpage

\subsection{Результаты эксперимента при переменном количестве потоков}

Зафиксируем количество разбиений $k=100$. Результаты эксперимента на рис. \ref{fig:6} показывают несколько вещей:

\begin{itemize}
    \item при увеличении потоков в $m$ раз, ускорение всегда происходит в $t < m$ раз; 
    \item несмотря на то, что количество потоков может быть больше, чем количество физических ядер, реального ускорения программы при количестве потоков $p$ большем, чем количество физических ядер -- не происходит.
\end{itemize}

\begin{figure}[h] 
\centering
\begin{tikzpicture}[scale = 1]
    \begin{axis}[ height = 0.4\paperheight
                , width = 0.7\paperwidth
                % , axis x line = bottom
                % , axis y line = middle
                , xtick = data
                , ytick = data
                , xmin = 0
                , xmax = 380000
                % , ymax = 0.43
                , xlabel = n
                , ylabel = micro seconds
                , yticklabel style = { /pgf/number format/fixed, /pgf/number format/precision=2 }
                ]

        \addplot [color = dark-blue, line width = 0.5mm, mark = *] 
            table {data/sort p=1 k=100.data};
        \addlegendentry{$p = 1$}

        \addplot [color = dark-green, line width = 0.5mm, mark = *]
            table {data/sort p=2 k=100.data};
        \addlegendentry{$p = 2$}

        \addplot [color = orange, line width = 0.5mm, mark = *]
            table {data/sort p=4 k=100.data};
        \addlegendentry{$p = 4$}

        \addplot [color = good-red, line width = 0.5mm, mark = *]
            table {data/sort p=6 k=100.data};
        \addlegendentry{$p = 6$}

        \addplot [color = dark-blue, line width = 0.5mm, mark = *] 
            table {data/sort p=8 k=100.data};
        \addlegendentry{$p = 8$}
    \end{axis}
\end{tikzpicture}
\caption{\label{fig:6}Ускорение при значении параметра $k = 100$ и разном количестве потоков}
\end{figure}

\section{Распараллеливание сортировки с помощью MPI}

Алгоритм распараллеливания с помощью MPI отличается от того же алгоритма на OpenMP. В алгоритме с MPI я фиксирую количество разбиений массива количеством используемых узлов. Каждый узел получает свою часть массива, которую сортирует и тогда нулевой узел собирает массив обратно и собирает отсортированный массив из полученных кусков. Представить такой алгоритм можно таким образом:

\begin{figure}[h] 
\centering
\begin{tikzpicture}[scale = 1]
    % {
    % [every join/.append style = {-Stealth, color = blue, line width = 0.3mm}, start chain = A]
    % \node [on chain, fill = white, draw, minimum height = 7mm, drop shadow] 
    %     (A) {1};
    % \node [on chain, fill = white, draw, minimum height = 7mm, drop shadow]
    %     (B) {2};
    % }
    \draw (-1, 0.5) node [draw, drop shadow, fill = white] {node 0:};
    \draw (0, 0) -- (6, 0) -- (6, 1) -- (0, 1) -- cycle;
    \draw (1, 0) -- (1, 1);
    \draw (2, 0) -- (2, 1);
    \draw (5, 0) -- (5, 1);
    \draw [-Stealth] (0.5, 0) arc (140:180:3) node [draw, drop shadow, fill = white, below] {node 0};
    \draw [-Stealth] (1.5, 0) arc (150:180:2) node [draw, drop shadow, fill = white, below] {node 1};
    \draw [-Stealth] (5.5, 0) arc (10:-20:3) node [fill = white, below] {...};
    \draw (3, -3) node [draw, drop shadow, fill = white] {node 0};
    \draw [-Stealth] (1.2, -1.6) arc (180:263:1.2);
    \draw [-Stealth] (-.2, -2.5) arc (240:270:4.9);
    \draw [-Stealth] (5.3, -1.9) arc (-30:-85:2);
    \draw (-2, -1.8) node {sort:};
\end{tikzpicture}
\caption{\label{fig:7}Алгоритм сортировки}
\end{figure}

Результаты распараллеливания представлены на рисунке ниже:

\begin{figure}[h] 
\centering
\begin{tikzpicture}[scale = 1]
    \begin{axis}[ height = 0.4\paperheight
                , width = 0.7\paperwidth
                % , axis x line = bottom
                % , axis y line = middle
                , xtick = data
                , ytick = data
                , xmin = 0
                , xmax = 380000
                % , ymax = 0.43
                , xlabel = n
                , ylabel = micro seconds
                , yticklabel style = { /pgf/number format/fixed, /pgf/number format/precision=2 }
                ]

        \addplot [color = dark-blue, line width = 0.5mm, mark = *] 
            table {data/mpi_sort k=1.data};
        \addlegendentry{$k = 1$}

        \addplot [color = dark-green, line width = 0.5mm, mark = *]
            table {data/mpi_sort k=2.data};
        \addlegendentry{$k = 2$}

        \addplot [color = orange, line width = 0.5mm, mark = *]
            table {data/mpi_sort k=4.data};
        \addlegendentry{$k = 4$}

        \addplot [color = good-red, line width = 0.5mm, mark = *]
            table {data/mpi_sort k=6.data};
        \addlegendentry{$k = 6$}

        \addplot [color = dark-blue, line width = 0.5mm, mark = *] 
            table {data/mpi_sort k=8.data};
        \addlegendentry{$k = 8$}

        \addplot [color = dark-green, line width = 0.5mm, mark = *] 
            table {data/mpi_sort k=12.data};
        \addlegendentry{$k = 12$}
    \end{axis}
\end{tikzpicture}
\caption{\label{fig:8}Время при распараллеливании MPI}
\end{figure}

\begin{table}[h]
\caption{Таблица средних ускорений}
\begin{tabularx}{\textwidth}{|c|c|}
\hline
Количество узлов & Среднее ускорение \\
\hline
$1$ & $1$ \\
$2$ & $4$ \\
$4$ & $11-16$ \\
$6$ & $38-40$ \\
$8$ & $66-69$ \\
$12$ & $122-134$ \\
\hline
\end{tabularx}
\end{table}

\newpage
\lstinputlisting[caption = {Исходный код программы для OpenMP}]{main.c}

\newpage
\lstinputlisting[caption = {Исходный код программы для MPI}]{mpi_main.c}

\end{document}
