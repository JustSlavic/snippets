\documentclass[12pt]{article}

\usepackage[T1]{fontenc}
\usepackage[utf8]{inputenc}
\usepackage[english, russian]{babel} 
\usepackage[scaled = 0.8]{PTMono} %monotype font
% \usepackage{paratype} %fonts
\usepackage{indentfirst} %indent after sections
\usepackage{xspace} % for additional spaces

\usepackage[left = 2cm, right = 1cm, top = 2cm, bottom = 2cm]{geometry}
\usepackage{lastpage}
\usepackage{setspace}
\onehalfspacing

\usepackage%
[pdftex,unicode,pdfborder={0 0 0},draft=false,%backref=page,
    hidelinks, % убрать, если хочется видеть ссылки: это
               % удобно в PDF файле, но не должно появиться на печати
    bookmarks=true,bookmarksnumbered=false,bookmarksopen=false]%
{hyperref}

\usepackage{tabularx}
\usepackage{ltablex}  % for centering inside cells of tables
\usepackage{pgfplots} % for histograms
\usepackage{xcolor}   % for defining new colors
\usepackage{amsmath}  % for split in equations
\usepackage{amssymb}  % for >= <=
\usepackage{commath}  % for dpd
\renewcommand {\le} {\leqslant}
\renewcommand {\ge} {\geqslant}

\definecolor{dark-blue}{rgb}{0.108, 0.3137, 0.70196}
\definecolor{dark-green}{rgb}{0.108, 0.7, 0.3}
\definecolor{good-red}{rgb}{0.7, 0.1, 0.3}
\definecolor{sky-blue}{rgb}{0.4, 0.6, 1}

\usepackage{tikz}
\usepackage{tikz-qtree}
\usetikzlibrary{scopes, 
  positioning, 
  chains, 
  arrows.meta, 
  shadows}

\usepackage{listings}
\lstset { basicstyle = \ttfamily
        , showspaces = false
        , showstringspaces = false
        , language = C++
        }

\begin{document}
\begin{centering}

\textbf{Отчёт}: <<Распараллеливание метода Рунге-Кутты 4 порядка>>\\

{Радько~В.\,А.} \\
\end{centering}

\section*{Решение дифференциального уравнения методом Рунге-Кутты 4 порядка}

\begin{figure}[h] 
\centering
\begin{tikzpicture}[scale = 1]
    \begin{axis}[ height = 0.4\paperheight
                , width = 0.7\paperwidth
                % , axis x line = bottom
                % , axis y line = middle
                % , xtick = data
                % , ytick = data
                % , xmin = 0
                % , xmax = 380000
                % , ymax = 0.43
                , xlabel = n
                , ylabel = micro seconds
                , yticklabel style = { /pgf/number format/fixed, /pgf/number format/precision=2 }
                ]

        \addplot [color = dark-blue, line width = 0.5mm, mark = *] 
            table {../data/p=1.data};
        \addlegendentry{$p = 1$}

        \addplot [color = dark-green, line width = 0.5mm, mark = *]
            table {../data/p=2.data};
        \addlegendentry{$p = 2$}

        \addplot [color = orange, line width = 0.5mm, mark = *]
            table {../data/p=4.data};
        \addlegendentry{$p = 4$}

        \addplot [color = good-red, line width = 0.5mm, mark = *]
            table {../data/p=6.data};
        \addlegendentry{$p = 6$}
    \end{axis}
\end{tikzpicture}
\caption{\label{fig:1}Время работы}
\end{figure}

\begin{figure}[h] 
\centering
\begin{tikzpicture}[scale = 1]
    \begin{axis}[ height = 0.4\paperheight
                , width = 0.7\paperwidth
                % , axis x line = bottom
                % , axis y line = middle
                % , xtick = data
                % , ytick = data
                % , xmin = 0
                % , xmax = 380000
                % , ymax = 0.43
                , xlabel = n
                , ylabel = micro seconds
                , yticklabel style = { /pgf/number format/fixed, /pgf/number format/precision=2 }
                ]

        \addplot [color = dark-blue, line width = 0.5mm, mark = *] 
            table {../data/speedup_p=1.data};
        \addlegendentry{$p = 1$}

        \addplot [color = dark-green, line width = 0.5mm, mark = *]
            table {../data/speedup_p=2.data};
        \addlegendentry{$p = 2$}

        \addplot [color = orange, line width = 0.5mm, mark = *]
            table {../data/speedup_p=4.data};
        \addlegendentry{$p = 4$}

        \addplot [color = good-red, line width = 0.5mm, mark = *]
            table {../data/speedup_p=6.data};
        \addlegendentry{$p = 6$}
    \end{axis}
\end{tikzpicture}
\caption{\label{fig:1}Ускорение}
\end{figure}

\newpage

\lstinputlisting[caption = {Исходный код программы для MPI}]{../main.c}

\end{document}
